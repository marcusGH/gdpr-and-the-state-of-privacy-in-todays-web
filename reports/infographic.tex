\documentclass[border=0mm]{article}
% 20cm x 24cm infographic
\usepackage[paperheight=24cm,paperwidth=20cm, left=.8cm, right=.8cm, top=1cm, bottom=0in, footskip=0in, ,heightrounded]{geometry}

\usepackage{graphicx} 
\usepackage{setspace} 
\usepackage[most]{tcolorbox} 
\usepackage{float} 
\usepackage{hyperref} 
\usepackage{xcolor} 
\usepackage[12pt]{moresize} 
\usepackage{array} 
\usepackage{tikz} 

% Lato font
\usepackage[default]{lato}
\usepackage{fontspec} 
\setmainfont[Ligatures=TeX]{Lato}

% Urban theme colours
\definecolor{col1}{RGB}{22,150,210}
\definecolor{col2}{RGB}{210,210,210}
\definecolor{col3}{RGB}{0,0,0}
\definecolor{col4}{RGB}{253,191,17}
\definecolor{col5}{RGB}{236,0,139}
\definecolor{col6}{RGB}{85,183,72}
\definecolor{col7}{RGB}{92,88,89}
\definecolor{col8}{RGB}{219,43,39}

\definecolor{dark}{RGB}{6,38,53}

\definecolor{bg}{RGB}{207,232,243}
\definecolor{bg2}{RGB}{245,245,245}
\definecolor{bg3}{RGB}{10,76,106}

% For the light-blue box on the right
\newtcolorbox{rightframe}[1][1]{
    enhanced,
    arc=0pt,
    outer arc=0pt,
    colback=bg,
    boxrule=0pt,
    #1
}

\newtcolorbox{references}[1][1]{
    enhanced,
    arc=0pt,
    outer arc=0pt,
    colback=col2,
    boxrule=0pt,
    #1
}

% Misc fixes
\setlength{\parindent}{0mm}
\newcolumntype{x}[1]{%
>{\raggedleft\hspace{0pt}}p{#1}}%

% \renewcommand{\baselinestretch}{.2}\selectfont

% Color adjustments
\pagecolor{bg2}

\begin{document}
% ------------------------------------ document start --------------------

\pagenumbering{gobble}

% -----------------------------------  Heading ---------------------------

% Good ol' readable tikz graphics
{\begin{tikzpicture}[remember picture,overlay]
    \node[yshift=-3cm] at (current page.north west)
    {\begin{tikzpicture}[remember picture, overlay]
        \draw[fill=bg3 ] (0,0) rectangle (\paperwidth,3cm);
        \node[anchor=east,xshift=.8\paperwidth,yshift=.11\paperwidth]{\color{col4}
        \HUGE \textbf{\uppercase{Do EU websites }} };
        \node[anchor=east,xshift=.96\paperwidth,yshift=.04\paperwidth]{\color{col4}
        \HUGE \textbf{\uppercase{care about your consent?}} };
    \end{tikzpicture}
    };
\end{tikzpicture}}

% To make the text not overlap with the heading
\vspace*{50pt}

% ---------------------------------- LEFT PANEL -------------------------

\begin{minipage}{0.5\textwidth}
\small
\vspace*{0.2cm}
We have all seen the annoying ``This website uses cookies'' pop-up when visiting a website.
Of the 13 million websites on the Internet, 10\% use such a cookie consent banner. Among
websites hosted in EU countries, more than 22\% have this banner.
This is due to the General Data Protection Regulation (GDPR), which
applies to all websites targeting EU residents
or are hosted in EU member states, that is, websites with a EU top-level-domain (TLD)
such as ``.dk''. To comply with the GDPR, these websites need to receive the users' consent
before using any cookies except those strictly necessary$^\ddagger$. One common use of cookies is delivering
personalised advertisements with Google AdSense, which almost 4\% of EU websites do.
However, a significant amount of traffic is directed towards websites which do this without
consent, as we see in the figure below. Yet, the GDPR has had a noticeable positive effect in reducing this number when compared
to websites hosted outside the EU.

\vspace*{.2cm}

\includegraphics[trim = {1cm 0 1cm 0},scale=.23]{outputs/gdpr-by-country.pdf}

{\small\linespread{.1}\selectfont\fontsize{8pt}{.1cm}\selectfont
\textbf{Source:} The HTTP Archive's web-page, technology and requests summaries.

\textbf{Notes:} The data is based on $\approx 10^9$ HTTP requests recorded by the HTTP Archive
    between January 1st 2023 and January 16th 2023. Only the $\approx 1.9 \cdot 10^8$ requests
    directed at websites with a EU TLD are considered, as these websites are
    obliged to comply with the GDPR.
    The HTTP Archive's technology dataset is used
    to determine which websites use Google AdSense and cookie consent banners. As Google AdSense
    requires consent to cookies for both personalised and non-personalised ads$^{\dagger}$,
    GDPR violations are counted by considering the proportion of HTTP requests to each EU TLD that
    is directed towards websites using Google AdSense without a cookie consent banner.
\par}


\begin{references}[width=1\textwidth,arc=10pt,auto outer arc]
\textbf{References:}

$(\dagger)$ \href{https://support.google.com/adsense/answer/9007336}{\color{dark} Google AdSense Help on Personalised and non-personalised ads}

$(\ddagger)$ \href{https://gdpr.eu/cookies/}{\color{dark} Cookies, the GDPR, and the ePrivacy Directive}
\end{references}



\end{minipage}% ------------------------------------------------------------------
\begin{minipage}{0.05\textwidth}
    \hspace{\fill}
\end{minipage}
% -----------------------  WEBSITES OF THE INTERNET ------------------------------
\begin{minipage}{0.45\textwidth}
    \vspace*{.2cm}

    \begin{rightframe}[width=1.03\textwidth,arc=10pt,auto outer arc]
    {\includegraphics[trim = {2.5cm 0 2cm 0}, scale=.33]{outputs/internet-waffle.pdf}
    \newline
    \vspace*{-.2cm}

    {\small\linespread{.2}\selectfont\fontsize{8pt}{.2cm}\selectfont
        \textbf{Source:} The HTTP Archive's web-page, technology and requests summaries.

        \textbf{Notes:} The data is based on $\approx 10^9$ HTTP requests and $\approx 10^7$
        web-pages tracked between January 1st 2023 and January 16th 2023.
        The HTTP Archive's technology dataset is used to determine web-page's use of
        Cookie consent banners and Google AdSense.  \par} }
    \end{rightframe}
\end{minipage}
\end{document}
